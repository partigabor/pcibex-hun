\documentclass[ruler]{CUP-JNL-LIN}%[ruler,draftrules]


%%%% Packages
\usepackage{graphicx}
\usepackage{multicol,multirow}
\usepackage{rotating}
\usepackage{appendix}
\usepackage{longtable}
\usepackage[sort&compress]{natbib}
\usepackage[breaklinks,colorlinks,allcolors=blue]{hyperref}
\ifpdf
  % pdflatex: breakurl not needed (and will warn)
\else
  \usepackage{breakurl} % for latex->dvips workflows
\fi

\articletype{RESEARCH ARTICLE}
\jname{Journal of Linguistics}
\jyear{2026}
\jvol{0}
%\jissue{1}
\jdoi{10.1017/S0022226721000XXX}

\copyrightline{\textcopyright\ The Author(s) 2021. This is an Open Access article,
distributed under the terms of the Creative Commons Attribution licence (http://creativecommons.org/licenses/by/4.0/),
which permits unrestricted re-use, distribution, and reproduction in any medium, provided the original work is properly cited.}

\begin{document}

\begin{Frontmatter}

\title[Short title]{Long title}

\author[1]{Gábor Parti}\corresp{\email{gabor.parti@connect.polyu.hk}}
\author[1]{Yu-Yin Hsu}

\authormark{Parti \& Hsu}

\address[1]{\orgname{The Hong Kong Polytechnic University}, \orgaddress{\street{11 Yuk Choi Rd, Hung Hom, Kowloon}, \country{Hong Kong SAR}}}
% \address[2]{\orgdiv{D\'epartement de math\'ematiques et de 	statistique}, \orgname{Univer\-sit\'e Laval, Pavillon Alexandre\-Vachon}, \orgaddress{\street{1045, av. de la M\'edecine}, \city{Qu\'ebec}, \postcode{Qc G1V 0A6}, \country{Canada}}}

\received{01 Jan 2026}
\revised{02 Jan 2026}
\accepted{03 Jan 2026}

\abstract{Abstract}

\keywords{a, b, c}



\end{Frontmatter}

\localtableofcontents

\section{Introduction}



Presidential scholars have long emphasised the role of the executive branch in
federal policymaking. Presidents develop policies formally through unilateral
action, but they also pursue their objectives in the legislative arena. Governors fill an
analogous role within their states. They manage the bureaucracy and help set the
policy agenda through speeches, calling special sessions or taking unilateral action.
I analyse factors that explain gubernatorial use of executive orders, and I consider how
these same executive orders influence statute adoption, using lesbian, gay, bisexual and
transgender (LGBT) employment protections as an illustrative case. Presidential scholars have long emphasised the role of the executive branch in
federal policymaking. Presidents develop policies formally through unilateral
action, but they also pursue their objectives in the legislative arena. Governors fill an
analogous role within their states. They manage the bureaucracy and help set the
policy agenda through speeches, calling special sessions or taking unilateral action.
I analyse factors that explain gubernatorial use of executive orders, and I consider how
these same executive orders influence statute adoption, using lesbian, gay, bisexual and
transgender (LGBT) employment protections as an illustrative case.

Presidential scholars have long emphasised the role of the executive branch in \cite{barclay03}
federal policymaking. Presidents develop policies formally through unilateral
action, but they also pursue their objectives in the legislative arena. Governors fill an
analogous role within their states. They manage the bureaucracy and help set the
policy agenda through speeches, calling special sessions or taking unilateral action.
I analyse factors that explain gubernatorial use of executive orders, and I consider how
these same executive orders influence statute adoption, using lesbian, gay, bisexual and
transgender (LGBT) employment protections as an illustrative case. Presidential scholars have long emphasised the role of the executive branch in
federal policymaking. Presidents develop policies formally through unilateral
action, but they also pursue their objectives in the legislative arena. Governors fill an
analogous role within their states. They manage the bureaucracy and help set the
policy agenda through speeches, calling special sessions or taking unilateral action.
I analyse factors that explain gubernatorial use of executive orders, and I consider how
these same executive orders influence statute adoption, using lesbian, gay, bisexual and
transgender (LGBT) employment protections as an illustrative case. They manage the bureaucracy and help set the
policy agenda through speeches, calling special sessions or taking unilateral action.
I analyse factors that explain gubernatorial use of executive orders, and I consider how
these same executive orders influence statute adoption, using lesbian, gay, bisexual and
transgender (LGBT) employment protections as an illustrative case.

Once data are disseminated, whatever contractual or other obligations are placed on those receiving \cite{berry90,berry99} the
data, the data are effectively out of a data providers' control. Data providers must be certain that the data
disseminated do not provide a risk of disclosure necessitating a reduction in the detail available, or they are
constrained to using a resource intensive auditing regime, and are likely to discover any data misuse only
after it has happened. Once data are disseminated, whatever contractual or other obligations are placed on those receiving the
data, the data are effectively out of a data providers' control. Data providers must be certain that the data
disseminated do not provide a risk of disclosure necessitating a reduction in the detail available, or they are
constrained to using a resource intensive auditing regime, and are likely to discover any data misuse only
after it has happened.  Presidential scholars have long emphasised the role of the executive branch in
federal policymaking. Presidents develop policies formally through unilateral
action, but they also pursue their objectives in the legislative arena. Governors fill an
analogous role within their states. They manage the bureaucracy and help set the
policy agenda through speeches, calling special sessions or taking unilateral action.
I analyse \nobreak factors that explain gubernatorial use of executive orders, and I consider how
these same \nobreak executive orders influence statute adoption, using lesbian, gay, bisexual and
transgender (LGBT) employment protections as an illustrative case. Presidential scholars have long emphasised the role of the executive branch in
federal policymaking. Presidents develop policies formally through unilateral
action, but they also pursue their objectives in the legislative arena. Governors fill an
analogous role within their states. 

Let $M$ be an $n$-dimensional smooth compact Riemannian manifold with boundary \hbox{$\Sigma=\partial M$}.  The Steklov
eigenvalue problem on $M$ consists in finding all numbers $\sigma\in\mathbb{R}$ for which there exists a nonzero function $u \in C^\infty(M)$, which solves
\begin{equation*}
    \begin{cases}
        \Delta u = 0 & \text{in  $M$,} \\
        \partial_\nu u = \sigma u & \text{on  $\Sigma$.}
    \end{cases}\vspace*{-2pt}
\end{equation*}
Here, $\Delta$ is the Laplacian induced from the Riemannian metric $g$ on $M$, and $\partial_\nu$
is the outward pointing normal derivative along the boundary $\Sigma$. The Steklov eigenvalues
form an unbounded increasing sequence
$0 = \sigma_0 \leq \sigma_1 \leq \sigma_2 \leq \dots \to \infty$, each of which is repeated according to its multiplicity.
Note that if $M$ is connected, then $\sigma_1>0$.


\section{Gubernatorial and presidential use of executive orders across the various states}

Presidents develop policies formally through unilateral
action, but they also pursue their objectives in the legislative arena. Governors fill an
analogous role within their states. They manage the bureaucracy and help set the
policy agenda through speeches, calling special sessions or taking unilateral action.
I analyse factors that explain gubernatorial use of executive orders, and I consider how
these same executive orders influence statute adoption, using lesbian, gay, bisexual and
transgender (LGBT) employment protections as an illustrative case.


\subsection{Presidential use of executive orders is largely consistent with expectations and previous literature}

The remainder of the findings is largely consistent \cite{berry98} with expectations and previous
literature. Diffusion plays a positive role on states adopting sexual orientation
protections; yet, it is not statistically significant in explaining the adoption of
transgender-inclusive statutes. As anticipated, legislatures are more likely to adopt
both forms of legislation in states where the citizens are more liberal.


\subsubsection{Third level heading with two line text style format with two line text style format  with two line text style format}

They manage the bureaucracy and help set the
policy agenda through speeches, calling special sessions or taking unilateral action.
I analyse factors that explain gubernatorial use of executive orders, and I consider how
these same executive orders influence statute adoption, using lesbian, gay, bisexual and
transgender (LGBT) employment protections as an illustrative case.

They manage the bureaucracy and help set the
policy agenda through speeches, calling special sessions or taking unilateral action.
I analyse factors that explain gubernatorial use of executive orders, and I consider how
these same executive orders influence statute adoption, using lesbian, gay, bisexual and
transgender (LGBT) employment protections as an illustrative case.

\section{Results}

\subsection{Determinants of executive orders}

The probability
of a state adopting legislation protecting \cite{boehmke09} sexual orientation increases by a
factor of 1.11 for a one-unit increase in Liberal Citizen Ideology, and the probability
increases by a factor of 2.24 for a five-unit increase in citizen ideology. This effect is
even more pronounced for transgender protections. A one-unit increase in Liberal
Citizen Ideology increases the likelihood of adoption by a factor of 1.20, and the
probability increases by a factor of 2.44 for a five-unit increase in citizen ideology.
The findings regarding the Evangelical population hint at a similar conclusion.

\section*{Estimation}

Using Multilevel Event History Analysis, with the state/year as the unit of analysis \cite{bolton15},
I evaluate the following:
\begin{enumerate}[2.]
\item The probability that a governor $i$ will issue an executive order protecting
LGBT employees in time $t$, given that no executive order is in place.

They manage the bureaucracy and help set the
policy agenda through speeches, calling special sessions or taking unilateral action.
\item The probability that the state legislature $i$ will adopt an LGBT-inclusive
employment nondiscrimination statute in time $t$, given that it has not
already done.
\end{enumerate}
Multilevel modelling accounts for these differences and within-state patterns of adoption seen
throughout the years \cite{brewer07}. The effect of determinants that
lead to successful statute adoption of LGBT protections share common elements,
but differ based on the type of protections added – sexual orientation versus gender
identity.
\begin{itemize}
\item The probability that a governor $i$ will issue an executive order protecting
LGBT employees in time $t$, given that no executive order is in place.

They manage the bureaucracy and help set the
policy agenda through speeches, calling special sessions or taking unilateral action.
\item The probability that the state legislature $i$ will adopt an LGBT-inclusive
employment nondiscrimination statute in time $t$, given that it has not
already done.
\end{itemize}
Multilevel modelling accounts for these differences and within-state patterns of adoption seen
throughout the years. The effect of determinants that
lead to successful statute adoption of LGBT protections share common elements,
but differ based on the type of protections added – sexual orientation versus gender
identity.

\begin{figure}[t]%
\FIG{\includegraphics[width=0.9\textwidth]{Fig}}
{\caption{This is a widefig. This is an example of long caption this is an example of long caption  this is an example of long caption this is an example of long caption}
\label{fig1}}
\end{figure}

\begin{figure}[t]%
\FIG{\includegraphics[width=0.9\textwidth]{Fig}}
{\caption{This is an example of short caption this is an example of short caption}
\label{fig2}}
\end{figure}

\begin{table}[t]
\TBL{\caption{Tables with short caption\label{tab2}}}
{\begin{fntable}
\tabcolsep=8pt
\begin{tabular*}{\textwidth}{@{\extracolsep{\fill}}lcccccc@{}}
\toprule
 \TCH{Projectile} & \TCH{Energy} & \TCH{$\sigma_{\mathit{calc}}$} & \TCH{$\sigma_{\mathit{expt}}$} &
\TCH{Energy} & \TCH{$\sigma_{\mathit{calc}}$} & \TCH{$\sigma_{\mathit{expt}}$} \\\midrule
\TCH{Element 3}&990 A &1168 &$1547\pm12$ &780 A &1166 &$1239\pm100$\\
{{Element 4}}&500 A &961 &$\hphantom{0}922\pm10$ &900 A &1268 &$1092\pm40\hphantom{0}$\\
\TCH{Element 3}&990 A &1168 &$1547\pm12$ &780 A &1166 &$1239\pm100$\\
{\TCH{Element 4}}&500 A &961 &$\hphantom{0}922\pm10$ &900 A &1268 &$1092\pm40\hphantom{0}$\\
\botrule
\end{tabular*}%
\end{fntable}}
\end{table}


\begin{table}[t]
\TBL{\caption{Tables which are too long to fit, should be written using the table environment as shown here\label{tab3}}}
{\begin{fntable}
\tabcolsep=8pt
\begin{tabular*}{\textwidth}{@{\extracolsep{\fill}}lcccccc@{}}
\toprule
 \TCH{Projectile} & \TCH{Energy} & \TCH{$\sigma_{\mathit{calc}}$} & \TCH{$\sigma_{\mathit{expt}}$} &
\TCH{Energy} & \TCH{$\sigma_{\mathit{calc}}$} & \TCH{$\sigma_{\mathit{expt}}$} \\\midrule
\TCH{Element 3}&990 A &1168 &$1547\pm12$ &780 A &1166 &$1239\pm100$\\
{{Element 4}}&500 A &961 &$\hphantom{0}922\pm10$ &900 A &1268 &$1092\pm40\hphantom{0}$\\
\TCH{Element 3}&990 A &1168 &$1547\pm12$ &780 A &1166 &$1239\pm100$\\
{\TCH{Element 4}}\footnotemark[a]&500 A &961 &$\hphantom{0}922\pm10$ &900 A &1268 &$1092\pm40\hphantom{0}$\\
\TCH{Element 3}&990 A &1168 &$1547\pm12$ &780 A &1166 &$1239\pm100$\\
{\TCH{Element 4}}&500 A &961 &$\hphantom{0}922\pm10$ &900 A &1268 &$1092\pm40\hphantom{0}$\\
\TCH{Element 3}&990 A &1168 &$1547\pm12$ &780 A &1166 &$1239\pm100$\\
{\TCH{Element 4}}&500 A &961 &$\hphantom{0}922\pm10$ &900 A &1268 &$1092\pm40\hphantom{0}$\\
\botrule
\end{tabular*}%
%\footnotetext[]{\textit{Note:} This is an example of table footnote this is an example of table footnote this is an example of table footnote this is an example of~table footnote this is an example of table footnote}
\footnotetext[a]{This is an example of table footnote}%
\end{fntable}
}
\end{table}


The final covariates analyse social factors that influence gubernatorial use of
executive orders. These results differ across the models. Diffusion is not statistically
significant for the sexual orientation model, but reaches conventional
statistical significance for the analysis of gender identity protections. This
tentatively suggests that governors are more likely to issue executive orders as
more neighbouring states add similar protections. Governors are more likely to
issue executive orders to protect \nobreak sexual \nobreak orientation when the states are more
liberal, and composed of fewer Evangelicals. Both terms reach conventional
statistical significance. However, this does not hold when the analysis turns to the
determinants of executive orders that protect gender identity. Citizen ideology
is not statistically significant and, counter to sexual orientation protections,
governors are more likely to issue executive orders when the Evangelical rate
increases. These discrepancies may be related to the changing strategies of
governors and LGBT advocates in later years, or it may be a reflection of the late
adopters that added protections through executive orders, i.e. the remaining
governors in states that were still ``at risk'' of adopting transgender protections
were in more socially conservative states. Both models show that governors are
more likely to issue protections later into the time frame, and the variance across
the states is statistically significant.


Diffusion plays an inconsistent role in policy adoption, but overall it seems that
the diffusion of pro-LGBT policies encourages the issuance of executive orders and
adoption of similar legislation. However, diffusion does not come up as statistically
significant and positive across the board, and thus caution should be taken when
examining its role in policy adoption. Governors used executive orders more
commonly to establish protections for sexual orientation, whereas legislation was
more prevalent for gender identity; therefore, this might explain why diffusion is
only statistically significant in those respective models. One possible explanation
for why diffusion of LGBT protections does not function as previous diffusion
studies suggest is because states consider several competing policies at once.


\begin{Backmatter}

\paragraph{Acknowledgments}
The authors also wish to thank Lucie Laporte-Devylder, Jordan Zlatev, Georgios
Stampoulidis, Joost Van de Weijer, Katie Hoemann, and two anonymous reviewers for their invaluable
feedback and comments on earlier versions of this paper.

\paragraph{Funding statement}
This study was funded by the Deutsche Forschungsgemeinschaft (DFG) through the Collaborative Research Center 1252 Prominence in Language, and by Mobility Grants from Division 7, Research Management, University of Cologne, which we gratefully acknowledge.

\paragraph{Data availability statement}
A statement about how to access data, code and other materials allowing users to understand, verify and replicate findings -- e.g. Replication data and code can be found in Harvard Dataverse: \url{https://doi.org/link}.



\begin{thebibliography}{99}

\bibitem[Arizona(2014)]{ariz14}
{Arizona Memory Project} (2014) Arizona Executive Orders, http://azmemory.azlibrary.gov/cdm/search/
collection/execorders (accessed 14 October 2014).

\bibitem[Barclay and Fisher(2003)]{barclay03}
{Barclay S and Fisher S} (2003) The states and the differing impetus for divergent paths on same-sex
marriage, 1990-2001. \textit{Policy Studies Journal} \textbf{31}, 331–352.

\bibitem[Berry and Berry(1990)]{berry90}
{Berry FS and Berry WD} (1990) State lottery adoptions as policy innovations: an event history analysis.
\textit{American Political Science Review} \textbf{84}, 395–415.

\bibitem[Berry and Berry(1999)]{berry99}
{Berry FS and Berry WD} (1999) Innovation and diffusion models in policy research. In Sabatier PA and
Weible CM (eds.), \textit{Theories of the Policy Process}. Berry and Berry-Boulder, CO: Westview Press, 307–360.

\bibitem[Berry et al.(1998)]{berry98}
{Berry WD, Ringquist EJ, Fording RC and Hanson RL} (1998) Measuring citizen and government ideology
in the American States, 1960-93. \textit{American Journal of Political Science} \textbf{42}, 327–348.

\bibitem[Boehmke(2009)]{boehmke09}
{Boehmke FJ} (2009) Approaches to modeling the adoption and diffusion of policies with multiple
components. \textit{State Politics \& Policy Quarterly} \textbf{9}, 229–252.

\bibitem[Bolton and Thrower(2015)]{bolton15}
{Bolton A and Thrower S} (2015) Legislative capacity and executive unilateralism. \textit{American Journal of
Political Science} \textbf{60}, 649–663.

\bibitem[Brewer(2007)]{brewer07}
{Brewer PR} (2007) \textit{Value War: Public Opinion and the Politics of Gay Rights}. New York: Rowman \&
Littlefield Publishers.

\bibitem[Burke(1992)]{burke92}
{Burke JP} (1992) \textit{The Institutional Presidency}. Baltimore: John Hopkins University Press.
Council of State Governments (2014) The Book of States 2014. Council of State Governments \url{http://dx.doi.org/10.1007/sxxxxx-xxx-xxxx-x}.

\end{thebibliography}


\end{Backmatter}

\end{document}
